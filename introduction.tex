\documentclass[12pt, letterpaper]{article}
\usepackage[utf8]{inputenc}
\usepackage{hyperref}
\usepackage{enumerate}
\usepackage{listings}
\usepackage{fancyvrb}
\usepackage[document]{ragged2e} % keep all left

\usepackage{minted} % yaml syntax highlighting

\newenvironment{markdown}%
    {\VerbatimEnvironment\begin{VerbatimOut}{tmp.markdown}}%
    {\end{VerbatimOut}%
        \immediate\write18{pandoc tmp.markdown -t latex -o tmp.tex}%
        \input{tmp.tex}}

\providecommand{\tightlist}{%
  \setlength{\itemsep}{0pt}\setlength{\parskip}{0pt}}

\newcommand*{\email}[1]{\href{mailto:#1}{\nolinkurl{#1}} } 

\title{Kubernetes Workshop \\{\small \href{https://creativecommons.org/licenses/by/4.0/}{CC BY 4.0} }  }
\author{Wojciech Barczynski (wbarczynski.pro@gmail.com)}
\date{}

\usepackage{pmboxdraw} % verbatim and unicode https://tex.stackexchange.com/questions/245777/latex-and-unicode-within-verbatim-environment

\begin{document}

\begin{titlepage}
\maketitle
\end{titlepage}

\tableofcontents
\pagebreak
\section{Prerequiments}

You need to feel good with Command Line Interface. You should understand what Docker is.

\begin{itemize}%
\item Workstation with Linux or OSX recommended.%
\item Software \begin{itemize}%
    \item Minikube
    \item Kubernetes CLI
    \item Docker
    \end{itemize}%
\item Tools \begin{itemize}%
    \item \href{https://stedolan.github.io/jq/}{jq}%
    \end{itemize}
\item Good to have \begin{itemize}%
    \item hub.docker.com account or alternative docker repository
    \end{itemize}
\end{itemize}

\subsection{HowTos}

\begin{markdown}
- [Install Minikube at kubernetes.io](https://kubernetes.io/docs/tasks/tools/install-minikube/)
- [Install Kubernetes CLI - kubectl at kubernetes.io](https://kubernetes.io/docs/tasks/tools/install-minikube/)
\end{markdown}

\subsection{Verify the setup}

\begin{verbatim}
$ minikube status
$ minikube start
\end{verbatim}

\begin{verbatim}
$ kubectl config use-context minikube
$ kubectl get nodes

NAME       STATUS   ROLES    AGE   VERSION
minikube   Ready    master    6d   v1.12.4
\end{verbatim}


\subsection{Kubernetes CLI Basics}

Let's learn first some basics regarding the \textit{kubectl}:

\begin{verbatim}
kubectl get <ARTIFACT>
kubectl describe <ARTIFACT>
\end{verbatim}

\bigskip
1. List the nodes

\begin{verbatim}
kubectl get nodes
\end{verbatim}

What the names of our nodes are? . . .

\bigskip
2. Learn about the node

\begin{verbatim}
kubectl get nodes minikube

# notice:
kubectl get no minikube
kubectl get node minikube
kubectl get node minikube -o wide
\end{verbatim}

\bigskip
3. Get more details: 

\begin{lstlisting}
kubectl describe nodes minikube
\end{lstlisting}

Note down:
\begin{itemize}
    \item Container Runtime Version: . . .
    \item  What the namespaces we have: . . .
    \item  Note down name of two pods:\begin{itemize}
        \item . . .
        \item . . .
     \end{itemize}
\end{itemize}

\bigskip
4. YAML and JSON output

\begin{verbatim}
kubectl get node minikube -o yaml
kubectl get node minikube -o json
\end{verbatim}

Use \textit{jq} to get the \textit{kubeletVersion}, write down below:
\\

   . . .

\bigskip
5. Notice jsonpath support

\begin{verbatim}
kubectl get node minikube \
  -o jsonpath="{.status.daemonEndpoints.kubeletEndpoint.Port}"
\end{verbatim}

\begin{verbatim}
kubectl get node minikube -o jsonpath="{.metadata.labels}"
\end{verbatim}

Write down a command with jsonpath to get information on how many CPU we have allocated to our minikube:
\\

   . . .


\bigskip
6. Select with labels

\begin{verbatim}
kubectl get nodes  -l 'kubernetes.io/hostname'=minikube
\end{verbatim}

Please find another label, you could select our node and run the command.


\bigskip
7. Select with labels
\begin{verbatim}
kubectl api-resources
\end{verbatim}

\bigskip
Hints

\begin{markdown}
- ```kubectx``` and ```kubens``` - https://github.com/ahmetb/kubectx
- ```alias k=kubectl``` or ```alias kb=kubectl```
\end{markdown}

%
%
%
\section{Kubectl configuration file}

Good to know. You can find there also your token, certificates, etc.

\bigskip
1. View ~/.kube/config

\bigskip
2. Find \textit{certificate-authority}

\bigskip
3. Note the main sections:

   . . .
   . . .
   . . .

\pagebreak
\section{Task at Hand}

We need to deploy \verb|intro-app| on kubernetes. Users will acess this application on \verb|my.app/echo|. Please help us, our next finanse round depens on it!

\section{What are the namespaces?}

\begin{verbatim}
$ kubectl get ns
$ kubectl get namespaces
\end{verbatim}

Notice:
\begin{itemize}
    \item you can create namespaces to better organize your components
    \item you might define resource restrictions per namespaces
    \item effect the name: \textit{$<$service-name$>$.$<$namespace-name$>$.svc.cluster.local}. We will talk about it later.
\end{itemize}

To change the selected namespace for our commands:

\begin{verbatim}
kubectl config set-context \
  $(kubectl config current-context) \
  --namespace <namespace-name>
\end{verbatim}

You can specify namesapce explicitly:

\begin{verbatim}
$ kubectl get pods --namespace=kube-system
$ kubectl get pods -n default
\end{verbatim}

%
%
%
\pagebreak
\section{Kubernetes deployments.yaml}

Let's get instances of our application running.

\bigskip
1. Get the deployment file \texttt{introduction/kube-deployment.yaml}:

\inputminted{yaml}{introduction/kube-deployment.yaml}

\smallskip
Notice: the postfix \verb|-deploy| is not the best practise, just to make it more explicit what-is-what during the training.

\smallskip
Notice:
\begin{itemize}
\item if your repo is private, you need to define \verb|imagePullSecrets|.
\item A force pulling image every time: \verb|imagePullPolicy: Always|, helpful during development, do not use in \textbf{production}.
\end{itemize}

\bigskip
2. Deploy:

\begin{verbatim}
#impreative:
$ kubectl create -f introduction/kube-deployment.yaml

#declarative:
$ kubectl apply -f introduction/kube-deployment.yaml
\end{verbatim}

To learn mode:\begin{itemize}
\item \href{https://kubernetes.io/docs/concepts/overview/object-management-kubectl/imperative-config/}{imperative config}
\item \href{https://kubernetes.io/docs/concepts/overview/object-management-kubectl/declarative-config/}{declarative config}
\end{itemize}

\bigskip
3. List deployments:

\begin{verbatim}
$ kubectl get deploy
$ kubectl get deployment
$ kubectl get deployments -o wide
\end{verbatim}

\bigskip
4. Check our deployment:

\begin{verbatim}
$ kubectl describe deploy <DEPLOYMENT_NAME>
\end{verbatim}

What the update strategy do we use?

   . . .

\bigskip
5. We should see pods: 

\begin{verbatim}
$ kubectl get po
$ kubectl get po -n default
\end{verbatim}

\begin{verbatim}
$ kubectl describe po <POD_NAME>
\end{verbatim}

\bigskip
6. Find the following information:
\begin{itemize}
    \item What is the IP of your app pod? . . .
    \item What is ReplicatSet? . . .
    \item Ready? . . .
    \item Restart Count? . . .
    \item Events? . . .
\end{itemize}

Notice: the next step are Liveness/Readiness probes.

%
%
%
\pagebreak
\section{Opening console in your container}

Imagine, we cannot reach our application. Let's check it within the running container.

\bigskip
1. Get the console:

\begin{verbatim}
kubectl exec -it intro-app-65d /bin/bash
\end{verbatim}

Notice: There is an ongoing discussion whether it is the "new" ssh.

\begin{verbatim}
kubectl exec -it intro-app-65d /bin/bash
\# printenv
\end{verbatim}

\bigskip
2. Add tool for debugging - curl:

\begin{verbatim}
apt-get update && apt-get install -qq curl
\end{verbatim}

\bigskip
3. Does it work?

\begin{verbatim}
# does it work?
curl 127.0.0.1

# can we get outside
curl -I wbarczynski.pl

# can we reach other services:
telnet kube-dns.kube-system 53
\end{verbatim}

\bigskip
5. Notice, the logs are going to stdout and Kubernetes lets us to see them:

\begin{verbatim}
kubectl logs intro-app-depl
\end{verbatim}

\bigskip
6. Yes, we fix the bug, let's clean up:

\begin{verbatim}
kubectl delete po intro-app-65db4-...
\end{verbatim}


\begin{verbatim}
# notice the name change
kubectl get po
\end{verbatim}

%
%
%
\section{Port-forwarding}

What If I told you, you can debug your app from your laptop.

\bigskip
1. Find the port our service listen, check the deployment file.

\bigskip
2. Setup the port forwarding:

\begin{verbatim}
kubectl proxy-forward <POD_NAME> 8080:<PORT_NUMBER>
\end{verbatim}

\bigskip
3. Use curl to query our app from your local console.

\bigskip
Let's learn about services and ingresses first, later 
we see how we can modify our deployment and update the application.

%
%
%
\pagebreak
\section{Kubernetes Service}

Our factory, I mean the deployment defines how we create our applications as pods. The service, how it is consumed.

\bigskip
1. Get the service file \texttt{introduction/kube-service.yaml}:

\inputminted{yaml}{introduction/kube-service.yaml}

\bigskip
2. Deploy:

\begin{verbatim}
$ kubectl create -f introduction/kube-service.yaml
$ kubectl apply -f introduction/kube-service.yaml
\end{verbatim}

\bigskip
3. Check:

\begin{verbatim}
$ kubectl get services
$ kubectl get svc
$ kubectl describe svc ...
\end{verbatim}

\bigskip
4. Find out the endpoint IP: . . .

\bigskip
5. We few types of services:

\begin{markdown}
- with ClusterIP
- ClusterIP: None
- LoadBalanced
\end{markdown}

\bigskip
6. Let's access it:

\begin{verbatim}
export SVC_PORT=$(kubectl get service intro-app-svc \
    --output='jsonpath="{.spec.ports[0].nodePort}"' \
    | tr -d '"')
curl  $(minikube ip):${SVC_PORT}
\end{verbatim}

Notice: on Azure, AWS, or Google, we would get the loadbalancer created and public IP assigned.

\bigskip
7. How it works?

\begin{verbatim}
$ kubectl exec -it intro-app-65db487447-lrhb9 /bin/bash

/# apt-get update && apt-get install dnsutils -qq
/# # resolve the same host names:
/# nslookup intro-app-svc
\end{verbatim}

\bigskip
8. How it works from other apps?

\begin{verbatim}
$ kubectl run -i --tty busybox --image=busybox -- sh
/# curl intro-app
/#
/# wget -O- intro-app-svc
/# wget -O- intro-app-svc.default
/# wget -O- intro-app-svc.default.svc
/# wget -O- kubernetes-dashboard.kube-system
\end{verbatim}

\bigskip
9. Delete the busybox deployment.

%
%
%
\section{Modyfing kubernetes deployment and service}

Avoid editing files on kubernetes, always modify a yaml and apply the changes.

\bigskip
1. Change the number of pods running to 2 with:

\begin{verbatim}
$ kubectl edit deploy
\end{verbatim}

\begin{verbatim}
$ kubectl get po
\end{verbatim}

\bigskip
2. Change the value of label \verb|me| to your name in the service definition.

\bigskip
3. Modify the \verb|depoyment.yml| to get 3 pods, use: \texttt{kubectl apply -f}

\bigskip
4. Add one more label to service.

\bigskip
5. What does happen if we add one more selector, apply it:

\inputminted{yaml}{introduction/exercise-broke-svc/kube-service-broke.yml}


Can we connect?

\begin{verbatim}
curl -I  $(minikube ip):${SVC_PORT}
\end{verbatim}

What has changed?
\begin{verbatim}
kubectl describe svc intro-app-svc
\end{verbatim}

\bigskip
Notice: {\large very very very common issue} :D

\bigskip
6. Fix our service.

%
%
%
\section{Updating service}

Let's update our app from the version \verb|1.0.0| to \verb|2.0.0|:

\bigskip
1. Change in the deployment file and apply changes.

\bigskip
2. You can also change it with set image:

\begin{verbatim}
$ kubectl set image deployment/<DEPLOYMENT_NAME> \
  <CONTAINER_NAME>=<DOCKER_IMAGE_NAME>:<VERSION>
\end{verbatim}

\bigskip
3. Change two times from \verb|1.0.0| to \verb|2.0.0| and back:

\begin{verbatim}
$ curl -I  $(minikube ip):${SVC_PORT}
\end{verbatim}

%
%
%
\pagebreak
\section{Kubernetes Ingress}

\smallskip
1. Install the ingress-controller --- \verb|traefik|:

\begin{verbatim}
$ kubectl apply -f  introduction/ingress-controller
\end{verbatim}

\bigskip
2. Apply the ingress configuration for our service - \verb|introduction/kube-ingress.yaml|:
\inputminted{yaml}{introduction/kube-ingress.yaml}

\bigskip
3. Check:

\begin{verbatim}
kubectl get ing
kubectl describe ing <YOUR_ING>
\end{verbatim}

\bigskip
4. Let's access our service as our customers would do:

\begin{verbatim}
$ export ING_CNTR_PORT=$(kubectl \
  get service traefik-ingress-service \
  -n kube-system \
  --output='jsonpath="{.spec.ports[0].nodePort}"' \
    | tr -d '"')

$ export MK_IP=$(minikube ip)

$ curl --header 'Host: my.app'  http://${MK_IP}:${ING_CNTR_PORT}/echo

\end{verbatim}

\bigskip
5. Not everything is CLI --- \verb|traefik| dashboard:

\begin{verbatim}
kubectl port-forward -n kube-system \
traefik-ingress-controller-s58cv \
8080:8080
\end{verbatim}

\section{Containers vs Pods}
Please answer the following questions:
\begin{itemize}
\item How many containers can a Pod has?
\item Do containers share disk?
\item Do container share port space?
\item What does \verb|1/1| mean in the output of \verb|kubectl get po|?
\end{itemize}

\bigskip
\bigskip

\section{Fail-over}
Let's see what happens when our application crashes.

\bigskip
1. Open console.

\bigskip
2. Force restart:

\begin{verbatim}
# should work
kill 1

# always works
kill -9 1
\end{verbatim}

Repeat 5 times. Observer the output from: \verb|kubectl get po|.

%
%
%
\section{How to debug}

Good to ship a minimum of debugging tools in your container, such as, curl or telnet.

Happy debugging path:

\begin{verbatim}
kubectl describe ing
kubectl describe svc
kubectl exec -it <> /bin/bash

# curl, telnet, ...
kubectl describe po <>
\end{verbatim}

\begin{verbatim}
kubectl logs <>
kubectl logs <> -f
kubectl logs <> --tail=100
\end{verbatim}

\begin{verbatim}
kubectl logs <YOUR_INGRESS_CONTROLER_POD>
\end{verbatim}

\begin{verbatim}
kubectl get events
\end{verbatim}

\bigskip
Notice: observability is a key, you should have at least monitoring.
%
%
%
\pagebreak
\section{Kubernetes configmap}

With configmaps, we can deliver values for environment variables or files. Let's change the page in our application:

\bigskip
1. Copy \verb|index.html|:

\begin{verbatim}
cp introduction/dockers/site-1.0.0/index.html index.html
\end{verbatim}

\bigskip
2. Add your name after the version number:

\begin{verbatim}
<html>
<h1>1.0.0-Maria</h1>
</html>
\end{verbatim}

\bigskip
3. Let's create a configmap:

\begin{verbatim}
kubectl create cm index-html --from-file index.html
\end{verbatim}

\bigskip
4. Check the commands:

\begin{verbatim}
kubectl describe cm index-html
kubectl get cm index-html -o yaml
kubectl get cm index-html -o json
\end{verbatim}

\pagebreak
5. Let's mount it:

\begin{minted}{yaml}
apiVersion: apps/v1
kind: Deployment
metadata:
  name: intro-app-deploy
  labels:
    app_deploy: intro-app
spec:
  replicas: 1
  selector:
    matchLabels:
      app: intro-app
  template:
    metadata:
      labels:
        app: intro-app
    spec:
      containers:
      - name: app
        image: wojciech11/api-status:1.0.0
        ports:
        - containerPort: 80
        volumeMounts:
        - mountPath: "/usr/share/nginx/html"
          name: "html-content"
      volumes:
      - name: html-content
        configMap:
          name: index-html
\end{minted}

\bigskip
5. Smoke test:

\begin{verbatim}

# 30288 is the port, the ingress-controller is listening:
curl --header 'Host: my.app'  "http://$(minikube ip):30288/echo"
\end{verbatim}

\bigskip
6. We can also set environment values, let's create new configmap:

\begin{verbatim}
kubectl create configmap intro-app-config \
  --from-literal=db.name=mydb
\end{verbatim}

\bigskip
7. .. and use it:

\begin{minted}{yaml}
        env:
        - name: DB_NAME
          valueFrom:
            configMapKeyRef:
              name: intro-app-config
              key: db.name
\end{minted}

\bigskip
8. Open a console in your pods and check whether the ENV variable is set:

\begin{verbatim}
\# printenv | grep DB_NAME
\end{verbatim}

\bigskip
Recomendation: Keep everything minimal.
%
%
%
\section{Kubernetes secret}

Secrets are very similar to configmaps. They provide minimal better security than configmaps.

\bigskip
1. Create a secret with database password:

\begin{verbatim}
kubectl create secret generic intro-app-secret \
  --from-literal="db.password=nomoresecrets"
\end{verbatim}

\bigskip
2. Bind it to environment variable in the deployment:

\begin{verbatim}
    env:
      - name: DB_PASSWORD
        valueFrom:
          secretKeyRef:
            name: intro-app-secret
            key: db.password
\end{verbatim}


\bigskip
3. Please deliver cert.crt to your application and mount it in \verb|/usr/secet|:

\begin{verbatim}
echo "CERT" > cert.crt
kubectl create secret generic intro-app-cert --from-file cert.crt
\end{verbatim}

%
\section{Kubernetes Persistent Volumes}

A persistence storage that survives your pod being deleted.

\bigskip
1. Storage class

\begin{verbatim}
kubectl get storageclasses

NAME                 PROVISIONER                AGE
standard (default)   k8s.io/minikube-hostpath   223d
\end{verbatim}

\begin{verbatim}
kubectl describe storageclasses standard

NAME                 PROVISIONER                AGE
standard (default)   k8s.io/minikube-hostpath   223d
\end{verbatim}

\bigskip
2. Persistence claim and Persistence volume

\inputminted[frame=single]{yaml}{introduction/persistentvolumes/pv.yaml}

\inputminted[frame=single]{yaml}{introduction/persistentvolumes/pvc.yaml}

\bigskip
3. Let's use it:

\inputminted[frame=single]{yaml}{introduction/persistentvolumes/kube-deployment.yaml}

\bigskip
4. Find where the mount point is on the host and create there file. Notice: \emph{minikube ssh}

\bigskip
5. Find the file on the pod with mounted volume.

%%
\section{Daemonset}
Why are good use cases for Daemonset? See our treafik ingress controller kubernetes yaml files.

%%
\section{Statefulsets}
What if we want to have a database on Kubernetes? Maybe we would like to have deterministic names. Statefulsets comes to rescue:

\bigskip
1. Simple DB:

\inputminted[frame=single]{yaml}{introduction/statefulsets/kube-statefulsets.yaml}

\bigskip
Note down what happens after:

\begin{verbatim}
kubectl scale --replicas=2 statefulset  intro-db
\end{verbatim}


\bigskip
2. What is a statefulset without a PV. Let's delete the previous statefulset and get a new one;

\inputminted[frame=single]{yaml}{introduction/statefulsets/kube-statefulsets-vct.yaml}

\bigskip
Scale it up and check in particular \emph{PV} and \emph{PVC}.

%
\section{Opinionated Configuration}
The configuration and the generation of the kubernetes files is a hot topic.

\begin{markdown}
1. envsubst or similar approaches
2. kustomize
3. Helm
\end{markdown}

\bigskip
1. envsubst or similar approach.

\begin{minted}{yaml}
apiVersion: extensions/v1beta1
kind: Ingress
metadata:
  name: my-extractor
  annotations:
    kubernetes.io/ingress.class: traefik
    traefik.ingress.kubernetes.io/request-modifier: "ReplacePathRegex: /my-app/(.*) /$1"
spec:
  rules:
  - host: ${HOST}
    http:
      paths:
        - path: /extract
          backend:
            serviceName: extractor
            servicePort: 80
\end{minted}

\begin{verbatim}
export HOST=
envsubst < my-k8s.tmpl.yaml > my-k8s.yaml
\end{verbatim}

\bigskip
2. kustomize - overlay
\begin{verbatim}
.
├── base
│   ├── kube-deployment.yaml
│   ├── kube-service.yaml
│   └── kustomization.yaml
├── dev
│   ├── image.yaml
│   ├── kustomization.yaml
│   └── scale.yaml
├── production
│   ├── image.yaml
│   ├── kustomization.yaml
│   └── scale.yaml
└── staging
    ├── image.yaml
    ├── kustomization.yaml
    └── scale.YAML
\end{verbatim}


\bigskip
3. Helm is aiming to become a package manager for Kubernetes.

%
%
\section{Exploring Namespace kube-system}

Let's look around what we have here.

\bigskip
1. Get the list of pods in namespace kube-sytem:

\begin{verbatim}
$ kubectl get po -n=kube-system
\end{verbatim}

Use \verb|kubectl describe po <pod-name> --namespace=kube-system| to find what the version is of:
\begin{itemize}
    \item kube-proxy: . . .
    \item apiserver: . . .
    \item coredns: . . .
\end{itemize}

\bigskip
2. Get the list of services:

\begin{verbatim}
$ kubectl get svc --namespace=kube-system
\end{verbatim}

Use \verb|kubernetes describe svc <svc-name> --namespace=kube-system| to find the endpoints for:
\begin{itemize}
\item kube-dns: . . .
\item kubernetes-dashboard: . . .
\end{itemize}

\bigskip
3. Logs:

\begin{verbatim}
$ kubectl logs coredns-c4c -n=kube-system
$ kubectl logs coredns-c4c -n=kube-system -f
$ kubectl logs coredns-c4c -n=kube-system --tail=10
\end{verbatim}

Please display logs of:\\
\verb|kube-apiserver|, \verb|kube-proxy|, \verb|kube-scheduler|, and \verb|etcd-minikube|.


Later, we will also cover events:\\
\texttt{kubectl get events -n=kube-system}.

\bigskip
4. Get the console:

\begin{verbatim}
$ kubectl exec -it kube-apiserver-minikube \
 /bin/sh -n=kube-system
\end{verbatim}

\bigskip
5. Kubernetes Dashboard:

\begin{verbatim}
# on normal deployment:
# $ kubernetes proxy
$ minikube dashboard
\end{verbatim}

\bigskip
6. Basic metrics:

\begin{verbatim}
minikube addons enable metrics-server

# wait 5 seconds
kubectl top nodes
kubectl top pods
\end{verbatim}

%
\section{Liveness/Readiness probes}
See \href{https://github.com/wojciech12/talk_zero_downtime_deployment_with_kubernetes}{https://github.com/wojciech12/talk\_zero\_downtime\_deployment\_with\_kubernetes}. Notes:
\bigskip
\bigskip
\bigskip

%
\section{Resource, Limits and QoS}
Notes:

\bigskip
\bigskip
\bigskip

%
\section{RBAC}
See \emph{introduction/ingress-controller/traefik-ingress-controller\_rbac.yaml} and the \emph{Service Account} definition for traefik. Notes:
\bigskip
\bigskip
\bigskip

%
\section{Outlook}
What could be the next steps in learning k8s.

\begin{markdown}

What you could learn next.

Next course *Immediate (Developer)*:

1. Liveness/Readiness probes
2. Monitoring with Prometheus
3. Resource and Limits, QoS for your pods, schedule policies
4. Statefulsets
5. DaemonSets
6. Taints and Tolerations
7. Node affinity 

*Observability plus* with Istio demo as what might the future be:

1. Monitoring
2. Logging
3. Traceability

*Advance (Developer)*:

1. Zero-downtime deployment strategies
2. Horizontal scaling (beta: vertical pod scaling for the pets)
3. Continuous Deployment and Integration
4. TravisCI and Gitlab

*Network and Security*:

1. RBAC deep dive
2. Networking - Internal Loadbalancing - https://kubernetes.io/docs/concepts/services-networking/
3. Restricting Egress/Ingress with [Network Policies](https://kubernetes.io/docs/concepts/services-networking/network-policies/#default-deny-all-egress-traffic)

*Kubernetes customization*

1. Write your first CRD
2. Operators
3. Plugins to kubectl

*CloudNative Ecosystem*

1. Observability: Prometheus stack
2. Observability: EFK
3. Observability: Tracing
4. Ingress Controllers: Traefik, ... , talk about standard and controller-specific annotations
5. Cert-manager
6. Operators for etcd and Vault
7. Kubeless

*More*

1. Istio
2. Operators for ...

*Optionals*

1. Google Kubernetes Engine - GKE
2. Azure Kubernetes Service -  AKS
3. Amazon Elastic Kubernetes - EKS

Trainings and Consultancy: \email{wbarczynski.pro@gmail.com}
\end{markdown}

\end{document}